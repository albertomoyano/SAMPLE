\documentclass{book}
\usepackage[
HomeHTMLFilename=index,% Filename of the homepage.
HTMLFilename={node-},% Filename prefix of other pages.
mathjax,%Use MathJax to display math.
xindy,
latexmk]{lwarp}
\usepackage{lipsum}

\usepackage[spanish]{babel}
\usepackage{csquotes}
\usepackage{graphicx}
\usepackage[backend=biber,backref=true,indexing=cite,citestyle=authoryear,style=authoryear]{biblatex}
%\usepackage[xindy,toc,numberedsection=nolabel]{glossaries}
\usepackage[xindy,acronym,sanitizesort,toc=true,nonumberlist]{glossaries}
\makenoidxglossaries

%hyperref siempre debe quedar último
\usepackage{hyperref}
% termina preámbulo predefinido

%comienza preámbulo dinámico desde la base de datos
\boolfalse{FileSectionNames}% If false, numbers the files.
\renewcommand{\linkhomename}{Inicio}
\renewcommand{\linkpreviousname}{Anterior}
\renewcommand{\linknextname}{Siguiente}
\renewcommand{\sidetocname}{Sumario}
\setcounter{tocdepth}{2}% Include subsections in the \TOC.
\setcounter{secnumdepth}{2}% Number down to subsections.
\setcounter{FileDepth}{1}% Split \HTML\ files at sections
\setcounter{FootnoteDepth}{1}
\booltrue{CombineHigherDepths}% Combine parts/chapters/sections
\setcounter{SideTOCDepth}{1}% Include subsections in the side\TOC
\HTMLTitle{Después de parir}% Overrides \title for the web page.
\HTMLAuthor{Alberto Moyano}% Sets the HTML meta author tag.
\HTMLLanguage{es-ES}% Sets the HTML meta language.
\HTMLDescription{Cada página de salida HTML debe tener su propia meta descripción HTML, que usualmente aparece en los resultados de búsqueda en la web.}% Generalmente, está limitada a alrededor de 150 caracteres de longitud y no debe incluir el carácter de comillas dobles ASCII.}% Sets the HTML meta description.
\HTMLFirstPageTop{\includegraphics[width=5cm]{./media/cover.png}}
%\HTMLPageTop{Después de parir}
\HTMLFirstPageBottom{Información de contacto y legales (\url{https://www.edicionesimagomundi.com/})}
\HTMLPageBottom{\LinkPrevious | \LinkNext}
\HTMLKeywords{clave1,clave2,clave3}
\HTMLAddMeta{book-title}{Después de parir}
\HTMLAddMeta{subtitle}{Que dolor}
\HTMLAddMeta{trans-title}{After birthing}
\HTMLAddMeta{edition}{Primera}
\HTMLAddMeta{publisher-name}{Ediciones Imago Mundi}
\HTMLAddMeta{publisher-loc}{Buenos Aires}
\HTMLAddMeta{isbn}{000-000-0000-000-0}
\HTMLAddMeta{pub-date}{2024/05/17}
\HTMLAddMeta{volume}{1}
\HTMLAddMeta{series-title}{Colección Bitácora Argentina}
\HTMLAddMeta{copyright-statement}{© 2023 Example Publisher}
\HTMLAddMeta{license-type}{Tipo de licencia: CC BY}
\HTMLAddMeta{license-text}{Creative Commons Attribution 4.0 International License}
\HTMLAddMeta{funding-source}{Fuente de financiamiento: National Science Foundation}
\HTMLAddMeta{abstract}{Cada página de salida HTML debe tener su propia meta descripción HTML, que usualmente aparece en los resultados de búsqueda en la web. Generalmente, está limitada a alrededor de 150 caracteres de longitud y no debe incluir el carácter de comillas dobles ASCII.}
\CSSFilename{imago_custom.css}

% fin del preambulo

% agregamos las referencias
% we added the references
\addbibresource{./files/gbTeXbib-SAMPLE.bib}

% agregamos el glosario
% we added the glossary

\newglossaryentry{@glo6-asimra}{
type = \acronymtype,
name         = {ASIMRA},
description  = {Asociación de Supervisores de la Industria Metalúrgica de la República Argentina},
first        = {Asociación de Supervisores de la Industria Metalúrgica de la República Argentina (ASIMRA)},
text         = {ASIMRA},
}
\newglossaryentry{@glo154-jng}{
type = \acronymtype,
name         = {JNG},
description  = {Junta Nacional de Granos},
first        = {Junta Nacional de Granos (JNG)},
text         = {JNG},
}
\newglossaryentry{@glo198-tai}{
name         = {Copiadora TAI Debrie.},
description  = {Copiadora óptica producida por la compañía francesa Debrie. Muchos archivos y laboratorios fílmicos utilizan la impresora TAI para duplicar películas de nitrato deterioradas y frágiles porque puede procesar película de archivo contraída y puede equiparse con una \gls{@glo199-vental}},
text         = {\textsc{copiadora tai debrie}},
}
\newglossaryentry{@glo201-ceheal}{
type = \acronymtype,
name         = {CEHEAL},
description  = {Centro de Estudios de Historia Económica Latinoamericana y Argentina},
first        = {Centro de Estudios de Historia Económica Latinoamericana y Argentina (CEHEAL)},
text         = {CEHEAL},
}
\newglossaryentry{@glo202-joseingenieros}{
name         = {José Ingenieros},
description  = {(nacido como Giuseppe Ingegnieri,​ Palermo, 24 de abril de 1877 - Buenos Aires, 31 de octubre de 1925) fue un médico, psiquiatra, psicólogo, criminólogo, farmacéutico, sociólogo, filósofo, masón, teósofo, ​escritor y docente ítaloargentino. Su libro \emph{Evolución de las ideas argentinas} marcó rumbos en el entendimiento del desarrollo histórico de Argentina como nación. Se destacó por su influencia entre los estudiantes que protagonizaron la Reforma Universitaria de 1918},
text         = {José Ingenieros},
}

\title{Después de parir}
\date{\today}
\begin{document}
\frontmatter

\maketitle

\tableofcontents

\chapter{Portada}

Resolver un diseño de página tal que aparezca la tapa y los datos de catalogación.

\chapter{Introducción}

\lipsum[1]

\gls{@glo6-asimra}

\gls{@glo201-ceheal}

\gls{@glo202-joseingenieros}

\emph{Nulla malesuada porttitor diam}. Donec felis erat, congue non, volutpat at, tincidunt tristique, libero. Vivamus viverra fermentum felis. Donec nonummy pellentesque ante. Phasellus adipiscing semper elit. Proin fermentum massa ac quam. Sed diam turpis, molestie vitae, placerat a, molestie nec, leo. Maecenas lacinia. Nam ipsum ligula, eleifend at, accumsan nec, suscipit a, ipsum. Morbi blandit ligula feugiat magna. Nunc eleifend consequat lorem. Sed lacinia nulla vitae enim. Pellentesque tincidunt purus vel magna. Integer non enim. Praesent euismod nunc eu purus. Donec bibendum quam in tellus. Nullam cursus pulvinar lectus. Donec et mi. Nam vulputate metus eu enim. Vestibulum pellentesque felis eu massa \parencite{@940-SHUMWAY1999}.\footnote{\emph{Nulla malesuada porttitor diam}. Donec felis erat, congue non, volutpat at, tincidunt tristique, libero. Vivamus viverra fermentum felis. Donec nonummy pellentesque ante. Phasellus adipiscing semper elit.}

\lipsum[1]

\mainmatter

\chapter{Mainmatter}
\label{mychapter}

\lipsum[1]

\section{Title}

Lorem ipsum dolor sit amet, consectetuer adipiscing elit. Ut purus elit, vestibulum ut, placerat ac, adipiscing vitae, felis. Curabitur dictum gravida mauris. Nam arcu libero, nonummy eget, consectetuer id, vulputate a, magna. Donec vehicula augue eu neque. Pellentesque habitant morbi tristique senectus et netus et malesuada fames ac turpis egestas. Mauris ut leo. Cras viverra metus rhoncus sem. Nulla et lectus vestibulum urna fringilla ultrices. Phasellus eu tellus sit amet tortor gravida placerat. Integer sapien est, iaculis in, pretium quis, viverra ac, nunc. Praesent eget sem vel leo ultrices bibendum. Aenean faucibus. Morbi dolor nulla, malesuada eu, pulvinar at, mollis ac, nulla. Curabitur auctor semper nulla. Donec varius orci eget risus. Duis nibh mi, congue eu, accumsan eleifend, sagittis quis, diam. Duis eget orci sit amet orci dignissim rutrum \parencite{@3070-TARKOVSKI1995}.\footnote{\lipsum[3]}

\lipsum[1]

\begin{figure}[!ht]
	\centering
	\includegraphics[width=\linewidth]{./media/bn-imagen1.png}
	\caption{Este es el epígrafe de la figura.}\label{figura1}
\end{figure}

\section{Title section}

\begin{enumerate}
	\item primero
	\item segundo
	\item tercero
\end{enumerate}

\lipsum[3]

\begin{quote}
	\enquote{Lorem ipsum dolor sit amet, consectetuer adipiscing elit. Ut purus elit, vestibulum ut, placerat ac, adipiscing vitae, felis. Curabitur dictum gravida mauris. Nam arcu libero, nonummy eget, consectetuer id, vulputate a, magna. Donec vehicula augue eu neque. Pellentesque habitant morbi tristique senectus et netus et malesuada fames ac turpis egestas}.
\end{quote}

\lipsum[5]

\section{Title section}

\lipsum[1]

\chapter{next meeting}
\label{mychapter2}

\section{Title section}

\lipsum[1]

\lipsum[1]

Como vemos en la figura \ref{figura1}, lorem ipsum dolor sit amet.\footnote{\lipsum[1]}

\lipsum[1]

\lipsum[1]

Como vemos en la figura \ref{figura1}, lorem ipsum dolor sit amet \parencite{@3159-FUNES2006}.\footnote{\lipsum[1]}

\lipsum[1]

\lipsum[2]

\section{Title section}

\lipsum[3]

\section{Title section}

\lipsum[1]

\subsection{Title section}

\lipsum[1]

\chapter{and next meeting}
As discussed in \ref{mychapter2}, lorem ipsum dolor sit amet.

\lipsum[2]

\section{Title section}

\lipsum[3]

\subsection{Title section}

\lipsum[1]

\backmatter

\ForceHTMLPage
\printnoidxglossary[title={Índice de siglas},type=\acronymtype]

\ForceHTMLPage
\printnoidxglossary[title={Glosario de términos}]

\ForceHTMLPage
\printbibliography[heading=bibintoc]

\chapter{Legales}

\lipsum[4]

\chapter{Colofón}

La composición tipográfica de este libro se realizó utilizando \href{https://github.com/albertomoyano/gbtexpublisher}{gbTeXpublisher.}

Las familias tipográficas utilizadas dentro del libro son: IBM Plex, una superfamilia de tipografía abierta, diseñada y desarrollada conceptualmente por Mike Abbink en IBM con colaboración de Bold Monday y Libertinus, bifurcación de la fuente Linux Libertine, diseñada para el texto del cuerpo y la lectura extendida.


\end{document}
