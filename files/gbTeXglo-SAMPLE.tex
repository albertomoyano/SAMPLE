
\newglossaryentry{@glo6-asimra}{
type = \acronymtype,
name         = {ASIMRA},
description  = {Asociación de Supervisores de la Industria Metalúrgica de la República Argentina},
first        = {Asociación de Supervisores de la Industria Metalúrgica de la República Argentina (ASIMRA)},
text         = {ASIMRA},
}
\newglossaryentry{@glo154-jng}{
type = \acronymtype,
name         = {JNG},
description  = {Junta Nacional de Granos},
first        = {Junta Nacional de Granos (JNG)},
text         = {JNG},
}
\newglossaryentry{@glo198-tai}{
name         = {Copiadora TAI Debrie.},
description  = {Copiadora óptica producida por la compañía francesa Debrie. Muchos archivos y laboratorios fílmicos utilizan la impresora TAI para duplicar películas de nitrato deterioradas y frágiles porque puede procesar película de archivo contraída y puede equiparse con una \gls{@glo199-vental}},
text         = {\textsc{copiadora tai debrie}},
}
\newglossaryentry{@glo201-ceheal}{
type = \acronymtype,
name         = {CEHEAL},
description  = {Centro de Estudios de Historia Económica Latinoamericana y Argentina},
first        = {Centro de Estudios de Historia Económica Latinoamericana y Argentina (CEHEAL)},
text         = {CEHEAL},
}
\newglossaryentry{@glo202-joseingenieros}{
name         = {José Ingenieros},
description  = {(nacido como Giuseppe Ingegnieri,​ Palermo, 24 de abril de 1877 - Buenos Aires, 31 de octubre de 1925) fue un médico, psiquiatra, psicólogo, criminólogo, farmacéutico, sociólogo, filósofo, masón, teósofo, ​escritor y docente ítaloargentino. Su libro \emph{Evolución de las ideas argentinas} marcó rumbos en el entendimiento del desarrollo histórico de Argentina como nación. Se destacó por su influencia entre los estudiantes que protagonizaron la Reforma Universitaria de 1918},
text         = {José Ingenieros},
}